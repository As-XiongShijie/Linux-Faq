\documentclass[fontset=windows]{article}
\usepackage[margin=1in]{geometry}%设置边距,符合Word设定
\usepackage{ctex}
\usepackage{setspace}
\usepackage{lipsum}
\usepackage{graphicx}%插入图片
\graphicspath{{Figures/}}%文章所用图片在当前目录下的 Figures目录
\usepackage{hyperref} % 对目录生成链接,注:该宏包可能与其他宏包冲突,故放在所有引用的宏包之后
\hypersetup{colorlinks = true,  % 将链接文字带颜色
	bookmarksopen = true, % 展开书签
	bookmarksnumbered = true, % 书签带章节编号
	pdftitle = This is a testfile for vscode, % 标题
	pdfauthor =Ali-loner} % 作者
\bibliographystyle{plain}% 参考文献引用格式
\newcommand{\upcite}[1]{\textsuperscript{\cite{#1}}}

\renewcommand{\contentsname}{\centerline{目录}} %经过设置word格式后,将目录标题居中


\title{\heiti\zihao{2}对于LaTeX的示范文章}  %这个地方是标题
\author{\songti 熊世洁}  %这个地方是作者
\date{2020.08.02}  %这个地方是写作的时间

\begin{document}
	\maketitle
	\thispagestyle{empty}

\begin{abstract}
	\lipsum[2]  %修改这个地方,就是修改的摘要里面的内容
\end{abstract}

\tableofcontents

\section{这个地方是文章内的小标题}
Hello world! Hello Ali! As shown in figure \ref{1}
\begin{figure}[htbp]
	\centering
	\includegraphics[scale=0.1]{Ali.jpg}
	\caption{this is Ali}
	\label{1}
\end{figure}


\section{这个地方是文章内的小标题,第二个哦}

这句话是测试能否进行引用及支持中文 \upcite{1}

\bibliography{books}
\end{document}