\documentclass[fontset=windows]{article}
\usepackage[margin=1in]{geometry}%设置边距,符合Word设定
\usepackage{ctex}
\usepackage{setspace}
\usepackage{lipsum}
\usepackage{graphicx}%插入图片
\graphicspath{{Figures/}}%文章所用图片在当前目录下的 Figures目录
\usepackage{hyperref} % 对目录生成链接,注:该宏包可能与其他宏包冲突,故放在所有引用的宏包之后
\hypersetup{colorlinks = true,  % 将链接文字带颜色
	bookmarksopen = true, % 展开书签
	bookmarksnumbered = true, % 书签带章节编号
	pdftitle = This is a testfile for vscode, % 标题
	pdfauthor =Ali-loner} % 作者
\bibliographystyle{plain}% 参考文献引用格式
\newcommand{\upcite}[1]{\textsuperscript{\cite{#1}}}

\renewcommand{\contentsname}{\centerline{目录}} %经过设置word格式后,将目录标题居中


\title{\heiti\zihao{2}作业二}  %这个地方是标题
\author{\songti 熊世洁}  %这个地方是作者
\date{2021.05.22}  %这个地方是写作的时间

\begin{document}
	\maketitle
	\thispagestyle{empty}

\begin{abstract}
【作业】某油罐区有一个地上钢质油罐,储存液体为航空煤油,容量均为6000m³,直径均为26m,高度为13m,采用固定式液下喷射泡沫灭火系统。已知每个泡沫产生器连接管在防火堤外的长度为12m,储罐壁到防火堤的距离为6m,防火堤外泡沫混合液总管长度
为500m。试进行泡沫灭火系统设计。%\lipsum[2]  %修改这个地方,就是修改的摘要里面的内容
\end{abstract}

\tableofcontents

\section{选择泡沫液——灭火剂采用氟蛋白泡沫液} 
当采用液下式喷射系统时,应采用氟蛋白,成膜氟蛋白或者水成膜泡沫液  \upcite{1}

\section{确定系统设计技术数据}
\begin{enumerate} % 有序列表
    \item 泡沫混合液供给强度\par
{\zihao{3} $q_g = 5.0(  L / ( min.m^2) $ }\par
    \item  连续供给时间\par
{\zihao{3}$t_1 = 40 (min)$}
\end{enumerate}

\section{计算储罐的燃烧面积}

\begin{equation}
{\zihao{5} x = \frac{\pi D^2}
    {4}
    =\frac{\pi \times 26^2}
    {4}
    =530.66\  m^2 }
\end{equation}

\section{高背压泡沫产生器及及泡沫喷射口设置数量}
\begin{enumerate} % 有序列表
    \item 所需泡沫混合液总量\par
    {\zihao{3} $Q_h = q_gA = 5.0 \times 530.66 = 2653.3 (  L / ( min) ) $}\par
    \item 泡沫喷射口的设置数量\par
    查规范,不少于两个  \upcite{2}
\end{enumerate}

{\zihao{3}选用3个PCY1350G  $Q_h = 1350 \times  2 = 2700 (  L / ( min) ) $}\par
每个泡沫产生器的额定混合液流量

{\zihao{3} $q_1 = \frac{1350}
    {60}
    =22.5 (  L / ( s) ) $}
    \begin{enumerate} % 有序列表
        \item 所需泡沫混合液总量\par
        {\zihao{3} $Q_h = q_gA = 5.0 \times 530.66 = 2653.3 (  L / ( min) ) $}\par
        \item 泡沫喷射口的设置数量\par
        查规范,不少于两个  \upcite{2}
    \end{enumerate}








\bibliography{books}
\end{document}
